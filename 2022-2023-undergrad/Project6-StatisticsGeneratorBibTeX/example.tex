\section{Future Work}
\label{section:futurework}
There are several very interesting extensions to this work which will provide ample opportunities for extensions to the software but also theoretical considerations. In this thesis we have developed an approach for the generation of random points based on our own methodology (see section~\ref{chapter:theoretical}) which uses the combination of a Voronoi polygon approach and distance relationships with polygon centroids to generate patterns of spatial points. One very positive aspect of the work is that there are several exciting directions for future work on RADIAN. The RADIAN tool as presented here in this thesis is an excellent starting point for some more indepth future research beyond the current scope of this thesis.

\textbf{Generation Methods:} There are many potential models which could be used to generate points exhibiting a realistic geographical distribution. Future work could consider the generation of random point datasets along road, rail or water networks corresponding to the input polygon. Using existing real-world datasets in this way is not new. For example, in~\citet{doi:10.1080/13658816.2018.1440563} the authors generate spatially explicit synthetic populations from global (census and GIS) data. \citet{10.1093/jamia/ocaa303} use openly available heart disease and diabetes datasets as well as the MIMIC-III diagnoses database\footnote{\url{https://physionet.org/content/mimiciii/1.4/}} to generate synthetic health data  for machine learning applications. We believe that alternative approaches to spatial data generation could take inspiration from the work on urban design and city architecture which has emerged over the last 100 years and continues to influence urban design and simulation~\citep{doi:10.1177/1420326X20976058,alonso2018cityscope}.

\textbf{Point rejections:} RADIAN currently generates output in GeoJSON files and PostgreSQL PostGIS dump files. With some additional work output could be generated in other popular spatial data formats including ESRI Shapefiles, Geopackage (GPKG), Well Known Text (WKT) or KML for use in Google's mapping products. The rejection ratio for the generation process is generally around 0.15, resulting in 6 points being rejected for every accepted point. Future work would include further improvement of the generation algorithm, with a focus on reducing the number of attempts necessary to create an acceptable point. This would also improve the overall runtime of the tool. This could be achieved by dynamically adjusting the acceptable bounds within which coordinates can be generated during the various multi-stage buffer-based generation phases. This could help reduce the number of points generated outside of the current buffer/polygon that would inevitably be rejected later on in the generation process. These types of considerations are important if RADIAN was developed and deployed as a web-based tool in the future. Then the overall runtimes would be a more criticial factor particularly as input polygon scale and $N$ grow larger. 
